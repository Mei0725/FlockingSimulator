% THIS DOCUMENT IS TAILORED TO REQUIREMENTS FOR SCIENTIFIC COMPUTING.  IT SHOULDN'T
% BE USED FOR NON-SCIENTIFIC COMPUTING PROJECTS
\documentclass[12pt]{article}

\usepackage{amsmath, mathtools}
\usepackage{amsfonts}
\usepackage{amssymb}
\usepackage{graphicx}
\usepackage{colortbl}
\usepackage{xr}
\usepackage{hyperref}
\usepackage{longtable}
\usepackage{xfrac}
\usepackage{tabularx}
\usepackage{float}
\usepackage{siunitx}
\usepackage{booktabs}
\usepackage{caption}
\usepackage{pdflscape}
\usepackage{afterpage}

\usepackage[round]{natbib}

%\usepackage{refcheck}

\setlength{\parindent}{0pt}

\hypersetup{
    bookmarks=true,         % show bookmarks bar?
      colorlinks=true,       % false: boxed links; true: colored links
    linkcolor=red,          % color of internal links (change box color with linkbordercolor)
    citecolor=green,        % color of links to bibliography
    filecolor=magenta,      % color of file links
    urlcolor=cyan           % color of external links
}

\input{../Comments.text}
\input{../Common.text}

% For easy change of table widths
\newcommand{\colZwidth}{1.0\textwidth}
\newcommand{\colAwidth}{0.13\textwidth}
\newcommand{\colBwidth}{0.82\textwidth}
\newcommand{\colCwidth}{0.1\textwidth}
\newcommand{\colDwidth}{0.05\textwidth}
\newcommand{\colEwidth}{0.8\textwidth}
\newcommand{\colFwidth}{0.17\textwidth}
\newcommand{\colGwidth}{0.5\textwidth}
\newcommand{\colHwidth}{0.28\textwidth}

% Used so that cross-references have a meaningful prefix
\newcounter{defnum} %Definition Number
\newcommand{\dthedefnum}{GD\thedefnum}
\newcommand{\dref}[1]{GD\ref{#1}}
\newcounter{datadefnum} %Datadefinition Number
\newcommand{\ddthedatadefnum}{DD\thedatadefnum}
\newcommand{\ddref}[1]{DD\ref{#1}}
\newcounter{theorynum} %Theory Number
\newcommand{\tthetheorynum}{TM\thetheorynum}
\newcommand{\tref}[1]{TM\ref{#1}}
\newcounter{tablenum} %Table Number
\newcommand{\tbthetablenum}{TB\thetablenum}
\newcommand{\tbref}[1]{TB\ref{#1}}
\newcounter{assumpnum} %Assumption Number
\newcommand{\atheassumpnum}{A\theassumpnum}
\newcommand{\aref}[1]{A\ref{#1}}
\newcounter{goalnum} %Goal Number
\newcommand{\gthegoalnum}{GS\thegoalnum}
\newcommand{\gsref}[1]{GS\ref{#1}}
\newcounter{instnum} %Instance Number
\newcommand{\itheinstnum}{IM\theinstnum}
\newcommand{\iref}[1]{IM\ref{#1}}
\newcounter{reqnum} %Requirement Number
\newcommand{\rthereqnum}{R\thereqnum}
\newcommand{\rref}[1]{R\ref{#1}}
\newcounter{nfrnum} %NFR Number
\newcommand{\rthenfrnum}{NFR\thenfrnum}
\newcommand{\nfrref}[1]{NFR\ref{#1}}
\newcounter{lcnum} %Likely change number
\newcommand{\lthelcnum}{LC\thelcnum}
\newcommand{\lcref}[1]{LC\ref{#1}}

\usepackage{fullpage}

\newcommand{\deftheory}[9][Not Applicable]
{
\newpage
\noindent \rule{\textwidth}{0.5mm}

\paragraph{RefName: } \textbf{#2} \phantomsection 
\label{#2}

\paragraph{Label:} #3

\noindent \rule{\textwidth}{0.5mm}

\paragraph{Equation:}

#4

\paragraph{Description:}

#5

\paragraph{Notes:}

#6

\paragraph{Source:}

#7

\paragraph{Ref.\ By:}

#8

\paragraph{Preconditions for \hyperref[#2]{#2}:}
\label{#2_precond}

#9

\paragraph{Derivation for \hyperref[#2]{#2}:}
\label{#2_deriv}

#1

\noindent \rule{\textwidth}{0.5mm}

}

\begin{document}

\title{Software Requirements Specification for \progname: Numerical Integration 
of Flocking Dynamics over Structured Terrains} 
\author{\authname}
\date{\today}
	
\maketitle

~\newpage

% \pagenumbering{roman}

\tableofcontents

~\newpage

\section*{Revision History}

\begin{tabularx}{\textwidth}{p{3cm}p{2cm}X}
\toprule {\bf Date} & {\bf Version} & {\bf Notes}\\
\midrule
2026/02/03 & 1.0 & Initialization of the SRS Document\\
\bottomrule
\end{tabularx}


~\newpage

\section{Reference Material}

This section records information for easy reference.

\subsection{Table of Units}

Throughout this document SI (Syst\`{e}me International d'Unit\'{e}s) is employed
as the unit system.  In addition to the basic units, several derived units are
used as described below.  For each unit, the symbol is given followed by a
description of the unit and the SI name.
~\newline

\renewcommand{\arraystretch}{1.2}
%\begin{table}[ht]
  \noindent \begin{tabular}{l l l} 
    \toprule		
    \textbf{symbol} & \textbf{unit} & \textbf{SI}\\
    \midrule 
    \si{\metre} & length & metre\\
    \si{\kilogram} & mass	& kilogram\\
    \si{\second} & time & second\\
    \bottomrule
  \end{tabular}
  %	\caption{Provide a caption}
%\end{table}

\subsection{Table of Symbols}

The table that follows summarizes the symbols used in this document along with
their units. The symbols are listed in alphabetical order.

\renewcommand{\arraystretch}{1.2}
%\noindent \begin{tabularx}{1.0\textwidth}{l l X}
\noindent \begin{longtable*}{l l p{12cm}} \toprule
\textbf{symbol} & \textbf{unit} & \textbf{description}\\
\midrule 
$A_C$ & \si[per-mode=symbol] {\square\metre} & coil surface area\\
$A_\text{in}$ & \si[per-mode=symbol] {\square\metre} & surface area over which
heat is transferred in\\ 
$\mathbf{a}_{i,t}$ & \si{\metre\per\second\squared} & The acceleration of agent \textit{i} at time t\\
$\mathbf{A}_\mathbf{t}$ & -- & The alignment metric of flocking at time t \\
$\mathbf{C}_\mathbf{t}$ & \si{\metre} & The flocking cohesion at time t\\
$\mathbf{F}_{i,t}$ & \si{\kilogram\metre\per\second\squared} & The forces acting on the agent \textit{i} at time t\\
$\mathbf{F}_{align,i,t}$ & \si{\kilogram\metre\per\second\squared} & The alignment forces acting on the agent \textit{i} at time t\\
$\mathbf{F}_{cohere,i,t}$ & \si{\kilogram\metre\per\second\squared} & The cohesion forces acting on the agent \textit{i} at time t\\
$\mathbf{F}_{goal,i,t}$ & \si{\kilogram\metre\per\second\squared} & The force in agent \textit{i} generated by the agent to move toward the goal\\
$\mathbf{F}_{obs,i,t}$ & \si{\kilogram\metre\per\second\squared} & The force in agent \textit{i} generated by the agent to avoid obstacles\\
$\mathbf{F}_{separate,i,t}$ & \si{\kilogram\metre\per\second\squared} & The separation forces acting on the agent \textit{i} at time t\\
$i$  & -- & Index of any agent in the flocking model \\
$k$  & -- & Index of one of the obstacle in the obstacle set \\
$k_a$  & \si{\kilogram\per\second} & The weighting coefficient of alignment forces \\
$k_c$  & \si{\kilogram\per\second\squared} & The weighting coefficient of cohesion forces \\
$k_g$  & \si{\kilogram\per\second\squared} & The weighting coefficient of goal seeking forces \\
$k_o$  & \si{\kilogram\metre\squared\per\second\squared} & The weighting coefficient of obstacle avoidance forces \\
$k_s$  & \si{\kilogram\metre\squared\per\second\squared} & The weighting coefficient of separation forces \\
$N$ & -- & The total number of agent in the flocking model \\
$\mathbf{m}$ & \si{\kilogram} & The mass of agent \\
$\mathbf{M}$ & - & The selected numerical integration method \\
$\mathcal{N}_i$ & -- & The set of agents excluding agent \textit{i} \\
$\mathcal{O}$ & -- & The set of obstacles \\
$\mathcal{O}_k$ & -- & One of the obstacle in the obstacle set \\
$\mathbf{p}_{g}$ & \si{\metre} & The position of goal point \\
$\mathbf{p}_{i,start}$ & \si{\metre} & The position of agent \textit{i} at start\\
$\mathbf{p}_{i,t}$ & \si{\metre} & The position of agent \textit{i} at time t\\
$\mathbf{p}_{k}$ & \si{\metre} & The position of obstacle \textit{k} \\
$\bar{\mathbf{p}}_\mathbf{t}$ & \si{\metre} & The position of flocking center at time t \\
$\mathbb{R}^2$ & -- & The two-dimensional Euclidean space \\
$\mathbf{r}_{k}$ & \si{\metre} & The radius of obstacle \textit{k} \\
$\Delta t$ & \si{\second} & The time step used by the numerical integration method \\
$\mathbf{v}_{i,t}$ & \si{\metre\per\second} & The velocity of agent \textit{i} at time t\\
$\bar{v}_\mathbf{t}$ & \si{\metre\per\second} & The mean speed of flocking at time t\\
\bottomrule
\end{longtable*}

\subsection{Abbreviations and Acronyms}

\renewcommand{\arraystretch}{1.2}
\begin{tabular}{l l} 
  \toprule		
  \textbf{symbol} & \textbf{description}\\
  \midrule 
  A & Assumption\\
  DD & Data Definition\\
  GD & General Definition\\
  GS & Goal Statement\\
  IM & Instance Model\\
  LC & Likely Change\\
  PS & Physical System Description\\
  R & Requirement\\
  SRS & Software Requirements Specification\\
  TM & Theoretical Model\\
  EEuler & Explicit Euler\\
  RK4 & Runge-Kutta 4\\
  SIEuler & Semi-Implicit Euler\\
  \bottomrule
\end{tabular}\\

\section{Introduction}

Flocking animation is widely used in computer animation, biological systems, 
and engineering simulations to generate stable and realistic collective motion 
trajectories. Since computers operate on discrete data, the choice of numerical 
integration method, in addition to the motion model of individual agents, has a 
significant impact on the accuracy, stability, and visual realism of flocking 
simulations.

The following section presents an overview of the Software Requirements 
Specification (SRS) for a system that simulates flocking behavior over complex 
terrains using different numerical integration methods. This section describes 
the purpose of the document, the scope of the requirements, the characteristics 
of the intended reader, and the organization of the document.

\subsection{Purpose of Document}

The primary purpose of this document is to record the requirements of 
FlockingSimulator. It provides a detailed description of the system goals, 
assumptions, theoretical models, definitions, and other model derivation 
information, enabling readers to fully understand and verify the purpose and 
scientific basis of FlockingSimulator. Except for system constraints, this SRS 
will remain abstract and focuses on describing what problem is being solved 
rather than how to solve it.

In addition, this document will serve as the starting point for subsequent 
development phases, including the design specification and the software 
verification and validation plan.

\subsection{Scope of Requirements} 

The scope of the requirements includes the simulation and visualization of 
flocking dynamics in a two-dimensional environment using multiple numerical 
integration methods. 

The scope of the system is limited to deterministic simulations, and does not 
include stochastic influences such as random forces or external noise. And the 
system does not provide boundary collision handling or response mechanisms, nor 
does it support real-time control of agents. Addi, due to the adoption of a 
two-dimensional environment model, the concept of structured terrains is 
restricted to static obstacles only, excluding more complex terrain properties 
such as height variations, gradients, or dynamic surfaces.

\subsection{Characteristics of Intended Reader} \label{sec_IntendedReader}

Readers are expected to have an undergraduate-level background in physics and 
mathematics, roughly equivalent to first-year university courses in calculus 
and classical mechanics. The users of FlockingSimulator system can have a lower 
level of expertise, as explained in~\ref{SecUserCharacteristics}.

\subsection{Organization of Document}

The organization of this document follows template (\citet{SmithAndLai2005, 
SmithEtAl2007, SmithAndKoothoor2016}), and sequentially includes a brief 
introduction, general system description, and specific system description. 
  
For readers who prefer a bottom-up approach, it is recommended to start with 
the instance models(~\ref{sec_instance}) to gradually explore its data 
definitions(~\ref{sec_datadef}), theoretical models(~\ref{sec_theoretical}), 
and related information.

\section{General System Description}

This section provides general information about the system.  It identifies the
interfaces between the system and its environment, describes the user
characteristics and lists the system constraints.

\subsection{System Context}

~\ref{Fig_SystemContext} shows the system context. A circle represents an 
external entity outside the software, the user in this case. A rectangle 
represents the software system itself (FlockingSimulator). Arrows are used to 
show the data flow between the system and its environment.

\begin{figure}[h!]
\begin{center}
 \includegraphics[width=0.6\textwidth]{SystemContextFigure}
\caption{System Context}
\label{Fig_SystemContext} 
\end{center}
\end{figure}

FlockingSimulator is mostly self-contained. The only external interaction is 
through the user interface. The responsibilities of the user and the system 
are as follows:

\begin{itemize}
\item User Responsibilities:
\begin{itemize}
\item Provide the input data to the system, ensuring no errors in the data 
entry.
\item Define a valid obstacle distribution.
\end{itemize}
\item \progname{} Responsibilities:
\begin{itemize}
\item Determine if the inputs satisfy the required physical and software 
constraints.
\item Calculate the required outputs, and visualize the outputs as line plots 
or animations.
\end{itemize}
\end{itemize}

\subsection{User Characteristics} \label{SecUserCharacteristics}

Users are expected to have knowledge of calculus at the first-year 
undergraduate level.

\subsection{System Constraints}

There are no system constraints.

\section{Specific System Description}

This section first presents the problem description, which gives a high-level
view of the problem to be solved.  This is followed by the solution characteristics
specification, which presents the assumptions, theories, definitions and finally
the instance models.

\subsection{Problem Description} \label{Sec_pd}

\progname{} is intended to simulate flocking behavior in environments with 
obstacles using multiple numerical integration methods.

\subsubsection{Terminology and  Definitions}

This subsection provides a list of terms that are used in the subsequent
sections and their meaning, with the purpose of reducing ambiguity and making it
easier to correctly understand the requirements:

\begin{itemize}

\item Agent: A single individual within the flock, governed by simple flocking 
motion rules.
\item Numerical Integration Method: A method used to compute agents' positions 
and velocities over discrete time steps.
\item Structured Terrain: In this project, structured terrain refers to a 2D 
environment containing static obstacles that influence agent movement.

\end{itemize}

\subsubsection{Physical System Description} \label{sec_phySystDescrip}

The physical system of \progname{}, as shown in Figure~\ref{Fig_phySystDescrip},
includes the following elements:

\begin{itemize}

\item[PS1:] A group of autonomously moving agents.

\item[PS2:] The environment in which the agents move.

\item[PS3:] Structured obstacles in the environment.

\item[PS4:] Force acting on agents, which is the sum of inter-agent forces, 
obstacle-avoidance forces, and goal-seeking forces.

\item[PS5:] Start point. It is an area, rather than a single point.

\item[PS6:] Global target.

\end{itemize}

\begin{figure}[h!]
\begin{center}
%\rotatebox{-90}
{
 \includegraphics[width=0.5\textwidth]{PhySystDescrip}
}
\caption{Physical System Description}
\label{Fig_phySystDescrip} 
\end{center}
\end{figure}

\subsubsection{Goal Statements}

Given the length of timestep, the parameters of collective modules, the number 
of agents, the numerical integration method, and the layout of obstacles, the 
goal statements are:

\begin{itemize}

\item[GS\refstepcounter{goalnum}\theassumpnum \label{GS_Simulation}:] 
Support simulation of flocking dynamics in structured environments 
with obstacles.

\item[GS\refstepcounter{goalnum}\theassumpnum \label{GS_Comparison}:] 
Enable comparison of different numerical integration methods to 
evaluate their effects on stability and collective behavior.

\item[GS\refstepcounter{goalnum}\theassumpnum \label{GS_Visualizaion}:] 
Provide visualization of flocking dynamics to facilitate analysis 
and understanding.

\end{itemize}

\subsection{Solution Characteristics Specification}

The instance models that govern \progname{} are presented in
Subsection~\ref{sec_instance}.  The information to understand the meaning of the
instance models and their derivation is also presented, so that the instance
models can be verified.

\subsubsection{Types}

The solution is a computational, agent-based simulation system that models 
flocking dynamics in a two-dimensional environment using discrete-time 
numerical integration.

% \subsubsection{Scope Decisions}

% \plt{This section is optional.}
% \subsubsection{Modelling Decisions}

% \plt{This section is optional.}

\subsubsection{Assumptions} \label{sec_assumpt}

This section simplifies the original problem and helps in developing the
theoretical model by filling in the missing information for the physical system.
The numbers given in the square brackets refer to the theoretical model [TM],
general definition [GD], data definition [DD], instance model [IM], or likely
change [LC], in which the respective assumption is used.

\begin{itemize}

\item[A\refstepcounter{assumpnum}\theassumpnum \label{A_agentMass}:] 
Agent-Mass-Unit: Each agent has a unit mass. The mass value is not physically 
meaningful and is used only to relate force and acceleration.
\item[A\refstepcounter{assumpnum}\theassumpnum \label{A_obsShape}:] 
Circular-Obstacles: All obstacles are circular in shape.
\item[A\refstepcounter{assumpnum}\theassumpnum \label{A_obsInfluence}:] 
Obstacle-Influence-Radius: The influence radius of an obstacle is twice its 
actual radius.
\item[A\refstepcounter{assumpnum}\theassumpnum \label{A_goalPosition}:] 
Goal-Position: The goal position is fixed at (500,500)(m).
\item[A\refstepcounter{assumpnum}\theassumpnum \label{A_startPosition}:] 
Start-Position: The agents' starting positions are randomly generated within 
a circle of radius 5m centered at (0, 0)(m).
\item[A\refstepcounter{assumpnum}\theassumpnum \label{A_similarAgent}:] 
Similar-Agent: All agents share identical intrinsic properties, including 
mass and motion dynamics. The same equations are used for computing position, 
velocity, and applied forces for every agent. The only distinction between 
agents is their initial position.

\end{itemize}

\subsubsection{Theoretical Models}\label{sec_theoretical}

This section focuses on the general equations and laws that \progname{} is based
on.

~\newline

\noindent
\begin{minipage}{\textwidth}
\renewcommand*{\arraystretch}{1.5}
\begin{tabular}{| p{\colAwidth} | p{\colBwidth}|}
\hline
\rowcolor[gray]{0.9}
RefName & TM\refstepcounter{theorynum}\thetheorynum \label{NewtonSecLaw}:NewtonSecLaw\\
\hline
Label & Newton's second law of motion \\
\hline
Equation & $\mathbf{a}=\frac{\mathbf{F}}{m}$\\
\hline
Description &
\textbf{F} is the total force acting on an individual agent($\frac{kg \cdot m}{s^2}$), \\
& \textit{m} is the agent's mass(kg),\\
& \textbf{a} is acceleration of the agent($\frac{m}{s^2}$). \\
\hline
Notes & 
It explains the relationship between the forces acting on an agent and its acceleration.\\
\hline
Source & \href{https://en.wikipedia.org/wiki/Newton\%27s_laws_of_motion\#Second_law}{Definition of Newton's second law of motion}\\
\hline
Ref.\ By & \iref{TimeStepAcc}\\
\hline
\end{tabular}
\end{minipage}\\

~\newline

\noindent
\begin{minipage}{\textwidth}
\renewcommand*{\arraystretch}{1.5}
\begin{tabular}{| p{\colAwidth} | p{\colBwidth}|}
\hline
\rowcolor[gray]{0.9}
RefName & TM\refstepcounter{theorynum}\thetheorynum \label{ConTimeDy}:ConTimeDy\\
\hline
Label & Continuous-time agent dynamics \\
\hline
Equation & $\frac{d\mathbf{p}}{dt}=\mathbf{v}$\\
\hline
Description &
\textit{d}\textbf{p} is the change in position(m), \\
& \textit{dt} is a change in time(s),\\
& \textbf{v} is the agent's velocity($\frac{m}{s}$), \\
\hline
Notes & 
The above equation provides the velocity formulation for continuous-time \\
& dynamics. In practical systems, velocity is computed in a discrete manner(XXX). \\
\hline
Source & \href{https://en.wikipedia.org/wiki/Velocity}{Definition of Velocity}\\
\hline
Ref.\ By & \iref{ExplicitEuler}, \iref{RungeKutta4}, \iref{SemiImplicitEuler}\\
\hline
\end{tabular}
\end{minipage}\\

~\newline

\subsubsection{General Definitions}\label{sec_gendef}

This section collects the laws and equations that will be used in building the
instance models.

~\newline

\noindent
\begin{minipage}{\textwidth}
\renewcommand*{\arraystretch}{1.5}
\begin{tabular}{| p{\colAwidth} | p{\colBwidth}|}
\hline
\rowcolor[gray]{0.9}
Number& GD\refstepcounter{defnum}\thedefnum \label{ForceCombination}:ForceCombination\\
\hline
Label & The forces acting on the agent \textit{i} at time t\\
\hline
SI Units& \si{\kilogram\metre\per\second\squared}\\
\hline
Equation & 
$\mathbf{F}_{i,t} = \mathbf{F}_{align,i,t} + \mathbf{F}_{cohere,i,t} + \mathbf{F}_{separate,i,t} + \mathbf{F}_{obs,i,t} + \mathbf{F}_{goal,i,t} $\\
\hline
Description &
This equation shows the detailed composition of the forces acting on each agent. \\
& $\mathbf{F}_{align,i,t}$ is the alignment force in agent \textit{i} within the group($\frac{kg \cdot m}{s^2}$).\\
& $\mathbf{F}_{cohere,i,t}$ is the cohesion force in agent \textit{i} within the group($\frac{kg \cdot m}{s^2}$).\\
& $\mathbf{F}_{separate,i,t}$ is the separation force in agent \textit{i} within the group($\frac{kg \cdot m}{s^2}$).\\
& $\mathbf{F}_{obs,i,t}$ is the force in agent \textit{i} generated by the agent to avoid obstacles($\frac{kg \cdot m}{s^2}$).\\
& $\mathbf{F}_{goal,i,t}$ is the force in agent \textit{i} generated by the agent to move toward the goal($\frac{kg \cdot m}{s^2}$).\\
\hline
Source & \href{https://en.wikipedia.org/wiki/Boids}{The basic rules of flocking behavior}\\
\hline
Ref.\ By & \iref{TimeStepAcc}\\
\hline
\end{tabular}
\end{minipage}\\

\subsubsection{Data Definitions}\label{sec_datadef}

This section collects and defines all the data needed to build the instance
models. The dimension of each quantity is also given.

~\newline

\noindent
\begin{minipage}{\textwidth}
\renewcommand*{\arraystretch}{1.5}
\begin{tabular}{| p{\colAwidth} | p{\colBwidth}|}
\hline
\rowcolor[gray]{0.9}
Number& DD\refstepcounter{datadefnum}\thedatadefnum \label{SingleObstacle}:singleObstacle\\
\hline
Label& The single obstacle.\\
\hline
Symbol & $\mathbf{p}_k, \mathbf{r}_k$\\
\hline
SI Units & \si{\metre}\\
\hline
Equation& $\mathcal{O}_k = 
\left\{ \mathbf{x} \in \mathbb{R}^2 \;\middle|\; \|\mathbf{x} - \mathbf{p}_k\| \le r_k \right\}$\\
\hline
Description & 
It is the parameters of a single obstacle, where obstacles are restricted to \\
& circular shapes(\aref{A_obsShape}). \\
& $\mathbf{p}_k$ is the center position,\\
& $\mathbf{r}_k$ is the obstacle radius.\\
\hline
Sources& -- \\
\hline
Ref.\ By & \ddref{ObstacleSet}, \ddref{ObsAvoidForce}\\
\hline
\end{tabular}
\end{minipage}\\
~\newline

\noindent
\begin{minipage}{\textwidth}
\renewcommand*{\arraystretch}{1.5}
\begin{tabular}{| p{\colAwidth} | p{\colBwidth}|}
\hline
\rowcolor[gray]{0.9}
Number& DD\refstepcounter{datadefnum}\thedatadefnum \label{ObstacleSet}:obstacleSet\\
\hline
Label& The set of obstacles.\\
\hline
Symbol & $\mathcal{O}_k$\\
\hline
SI Units & \si{\metre}\\
\hline
Equation& $\{\mathcal{O}_k\}_{k=1}^M \subset \mathbb{R}^2$\\
\hline
Description & 
It is a set of \textit{M} obstacles, where M is specified by the user, \\
& with each obstacle having user-defined position and size.\\
& $\mathbb{R}^2$ is the two-dimensional Euclidean space.\\
\hline
Sources& -- \\
\hline
Ref.\ By & \ddref{ObsAvoidForce}\\
\hline
\end{tabular}
\end{minipage}\\

~\newline

\noindent
\begin{minipage}{\textwidth}
\renewcommand*{\arraystretch}{1.5}
\begin{tabular}{| p{\colAwidth} | p{\colBwidth}|}
\hline
\rowcolor[gray]{0.9}
Number& DD\refstepcounter{datadefnum}\thedatadefnum \label{StartPosition}:startPosition\\
\hline
Label& The position of start point for agent \textit{i}.\\
\hline
Symbol &$\mathbf{p}_{i,start}$\\
\hline
SI Units & \si{\metre}\\
\hline
Equation& $\mathbf{p}_{i,start} \in \left\{ \mathbf{x} \in \mathbb{R}^2 \;\middle|\; \|\mathbf{x} - 0\| \le 5 \right\}$\\
\hline
Description & 
It is the position of start point for agent \textit{i}.\\
& $\mathbb{R}^2$ is the two-dimensional Euclidean space.\\
& It will randomly be generated within a circle of radius 5m centered at (0, 0)(m).\\
\hline
Sources& -- \\
\hline
Ref.\ By & \iref{ExplicitEuler}, \iref{RungeKutta4}, \iref{SemiImplicitEuler}\\
\hline
\end{tabular}
\end{minipage}\\

~\newline

\noindent
\begin{minipage}{\textwidth}
\renewcommand*{\arraystretch}{1.5}
\begin{tabular}{| p{\colAwidth} | p{\colBwidth}|}
\hline
\rowcolor[gray]{0.9}
Number& DD\refstepcounter{datadefnum}\thedatadefnum \label{AlignForce}:alignForce\\
\hline
Label& The alignment force in agent \textit{i} within the group.\\
\hline
Symbol &$\mathbf{F}_{align,i,t}$\\
\hline
SI Units & \si{\kilogram\metre\per\second\squared}\\
\hline
Equation& $\mathbf{F}_{align,i,t} =
k_a\left(\frac{1}{|\mathcal{N}_i|}\sum_{j \in \mathcal{N}_i} \mathbf{v}_{j,t}-\mathbf{v}_{i,t}\right)$\\
\hline
Description & 
It ensures that agents in the flocking system maintain similar velocities.\\
& $k_a$ is the corresponding weighting coefficient($\frac{kg}{s}$),\\
& $\mathcal{N}_i$ is the set of other agents in the group,\\
& $\mathbf{v}_{i,t}$ is the agent \textit{i}'s velocity($\frac{m}{s}$).\\
\hline
Sources& \href{https://en.wikipedia.org/wiki/Boids}{Definition of Alignment Force}\\
\hline
Ref.\ By & \dref{ForceCombination}\\
\hline
\end{tabular}
\end{minipage}\\

~\newline

\noindent
\begin{minipage}{\textwidth}
\renewcommand*{\arraystretch}{1.5}
\begin{tabular}{| p{\colAwidth} | p{\colBwidth}|}
\hline
\rowcolor[gray]{0.9}
Number& DD\refstepcounter{datadefnum}\thedatadefnum \label{CohereForce}:cohereForce\\
\hline
Label& The cohesion force in agent \textit{i} within the group.\\
\hline
Symbol &$\mathbf{F}_{cohere,i,t}$\\
\hline
SI Units & \si{\kilogram\metre\per\second\squared}\\
\hline
Equation& $\mathbf{F}_{cohere,i,t}
= k_c\left(\frac{1}{|\mathcal{N}_i|}\sum_{j \in \mathcal{N}_i} \mathbf{p}_{j,t}-\mathbf{p}_{i,t}\right)$\\
\hline
Description & 
It ensures that agents in the flocking system do not drift too far away from other agents.\\
& $k_c$ is the corresponding weighting coefficient($\frac{kg}{s^2}$),\\
& $\mathcal{N}_i$ is the set of other agents in the group,\\
& $\mathbf{p}_{i,t}$ is the agent \textit{i}'s position(m).\\
\hline
Sources& \href{https://en.wikipedia.org/wiki/Boids}{Definition of Cohesion Force}\\
\hline
Ref.\ By & \dref{ForceCombination}\\
\hline
\end{tabular}
\end{minipage}\\

~\newline

\noindent
\begin{minipage}{\textwidth}
\renewcommand*{\arraystretch}{1.5}
\begin{tabular}{| p{\colAwidth} | p{\colBwidth}|}
\hline
\rowcolor[gray]{0.9}
Number& DD\refstepcounter{datadefnum}\thedatadefnum \label{SeparateForce}:separateForce\\
\hline
Label& The separation force in agent \textit{i} within the group.\\
\hline
Symbol &$\mathbf{F}_{separate,i,t}$\\
\hline
SI Units & \si{\kilogram\metre\per\second\squared}\\
\hline
Equation& $\mathbf{F}_{separate,i,t}
= k_s\sum_{j \in \mathcal{N}_i}\frac{\mathbf{p}_{i,t}-\mathbf{p}_{j,t}}{\|\mathbf{p}_{i,t} - \mathbf{p}_{j,t}\|^2}$\\
\hline
Description & 
It ensures that agents in the flocking system do not get too close to other agents.\\
& $k_s$ is the corresponding weighting coefficient($\frac{kg \cdot m^2}{s^2}$),\\
& $\mathcal{N}_i$ is the set of other agents in the group,\\
& $\mathbf{p}_{i,t}$ is the agent \textit{i}'s position(m).\\
\hline
Sources& \href{https://en.wikipedia.org/wiki/Boids}{Definition of Separation Force}\\
\hline
Ref.\ By & \dref{ForceCombination}\\
\hline
\end{tabular}
\end{minipage}\\

~\newline

\noindent
\begin{minipage}{\textwidth}
\renewcommand*{\arraystretch}{1.5}
\begin{tabular}{| p{\colAwidth} | p{\colBwidth}|}
\hline
\rowcolor[gray]{0.9}
Number& DD\refstepcounter{datadefnum}\thedatadefnum \label{ObsAvoidForce}:obsAvoidForce\\
\hline
Label& The obstacle avoidance force for agent \textit{i}.\\
\hline
Symbol &$\mathbf{F}_{obs,i,t}$\\
\hline
SI Units & \si{\kilogram\metre\per\second\squared}\\
\hline
Equation& $\mathbf{F}_{obs,i,t} =
\sum_{k \in \mathcal{O}}k_o \left( \frac{1}{\|\mathbf{p}_{i,t} - \mathbf{p}_k\|} - \frac{1}{r_k} \right)
\frac{\mathbf{p}_{i,t} - \mathbf{p}_k}{\|\mathbf{p}_{i,t} - \mathbf{p}_k\|}$\\
\hline
Description & 
It helps agent \textit{i} avoid collisions with obstacles.\\
& $k_o$ is the corresponding weighting coefficient($\frac{kg \cdot m^2}{s^2}$),\\
& $\mathcal{O}$ is the set of obstacles,\\
& $\mathbf{p}_{i,t}$ is the agent \textit{i}'s position(m).\\
& $\mathbf{p}_k$ is the center position of obstacle \textit{k}(m).\\
& $\mathbf{r}_k$ is the influence radius of obstacle \textit{k}(m).\\
& In order for this formula to work, it is assumed that all obstacles have a fixed influence radius\\
& (\aref{A_obsInfluence}).\\
\hline
Sources& \href{https://ace.ewapub.com/article/view/4237}{Common Obstacle Avoidance Force}\\
\hline
Ref.\ By & \dref{ForceCombination}\\
\hline
\end{tabular}
\end{minipage}\\

~\newline

\noindent
\begin{minipage}{\textwidth}
\renewcommand*{\arraystretch}{1.5}
\begin{tabular}{| p{\colAwidth} | p{\colBwidth}|}
\hline
\rowcolor[gray]{0.9}
Number& DD\refstepcounter{datadefnum}\thedatadefnum \label{GoalSeekForce}:goalSeekForce\\
\hline
Label& The goal seeking force for agent \textit{i}.\\
\hline
Symbol &$\mathbf{F}_{goal,i,t}$\\
\hline
SI Units & \si{\kilogram\metre\per\second\squared}\\
\hline
Equation& $\mathbf{F}_{goal,i,t} =
k_g \, \frac{\mathbf{p}_g - \mathbf{p}_{i,t}}{\|\mathbf{p}_g - \mathbf{p}_{i,t}\|}$\\
\hline
Description & 
It helps agent \textit{i} move toward the goal.\\
& $k_g$ is the corresponding weighting coefficient($\frac{kg \cdot m}{s^2}$),\\
& $\mathbf{p}_{i,t}$ is the agent \textit{i}'s position(m).\\
& $\mathbf{p}_g$ is the goal's position(m)(\aref{A_goalPosition}).\\
\hline
Sources& --\\
\hline
Ref.\ By &  \dref{ForceCombination}\\
\hline
\end{tabular}
\end{minipage}\\

~\newline

\noindent
\begin{minipage}{\textwidth}
\renewcommand*{\arraystretch}{1.5}
\begin{tabular}{| p{\colAwidth} | p{\colBwidth}|}
\hline
\rowcolor[gray]{0.9}
Number& DD\refstepcounter{datadefnum}\thedatadefnum \label{MeanSpeed}:meanSpeed\\
\hline
Label& The mean speed of flocking at time t.\\
\hline
Symbol &$\bar{v}_\mathbf{t}$\\
\hline
SI Units & \si{\metre\per\second}\\
\hline
Equation& $\bar{v}_\mathbf{t} = \frac{1}{N} \sum_{i=1}^{N} \|\mathbf{v}_{i,t}\|$\\
\hline
Description & 
It is one of the monitoring metrics of the flocking state, the average velocity \\
& of all agents.\\
& $N$ is the number of agents,\\
& $\mathbf{v}_{i,t}$ is the agent \textit{i}'s velocity at time t($\frac{m}{s}$).\\
\hline
Sources& \href{https://en.wikipedia.org/wiki/Flocking}{Definition of Flocking}\\
\hline
Ref.\ By & \rref{R_FlockingStatus}\\
\hline
\end{tabular}
\end{minipage}\\

~\newline

\noindent
\begin{minipage}{\textwidth}
\renewcommand*{\arraystretch}{1.5}
\begin{tabular}{| p{\colAwidth} | p{\colBwidth}|}
\hline
\rowcolor[gray]{0.9}
Number& DD\refstepcounter{datadefnum}\thedatadefnum \label{AlignMetric}:alignMetric\\
\hline
Label& The alignment metric of flocking at time t.\\
\hline
Symbol &$\mathbf{A}_\mathbf{t}$\\
\hline
SI Units & -- \\
\hline
Equation& $\mathbf{A}_\mathbf{t} = 
\|\frac{1}{N} \sum_{i=1}^{N} \frac{\mathbf{v}_{i,t}}{\|\mathbf{v}_{i,t}\|}\|$\\
\hline
Description & 
It is one of the monitoring metrics of the flocking state, the alignment metric \\
& of all agents.\\
& $N$ is the number of agents,\\
& $\mathbf{v}_{i,t}$ is the agent \textit{i}'s velocity at time t($\frac{m}{s}$).\\
\hline
Sources& \href{https://en.wikipedia.org/wiki/Flocking}{Definition of Flocking Alignment}\\
\hline
Ref.\ By & \rref{R_FlockingStatus}\\
\hline
\end{tabular}
\end{minipage}\\

~\newline

\noindent
\begin{minipage}{\textwidth}
\renewcommand*{\arraystretch}{1.5}
\begin{tabular}{| p{\colAwidth} | p{\colBwidth}|}
\hline
\rowcolor[gray]{0.9}
Number& DD\refstepcounter{datadefnum}\thedatadefnum \label{FlockingCenter}:flockingCenter\\
\hline
Label& The position of flocking center at time t.\\
\hline
Symbol &$\bar{\mathbf{p}}_\mathbf{t}$\\
\hline
SI Units & \si{\metre}\\
\hline
Equation& $\bar{\mathbf{p}}_\mathbf{t} = \frac{1}{N} \sum_{i=1}^{N} \mathbf{p}_{i,t}$\\
\hline
Description & 
It is the center of all agents.\\
& $N$ is the number of agents,\\
& $\mathbf{p}_{i,t}$ is the agent \textit{i}'s position at time t(m).\\
\hline
Sources& --\\
\hline
Ref.\ By & \ddref{FlockingCohesion}\\
\hline
\end{tabular}
\end{minipage}\\

~\newline

\noindent
\begin{minipage}{\textwidth}
\renewcommand*{\arraystretch}{1.5}
\begin{tabular}{| p{\colAwidth} | p{\colBwidth}|}
\hline
\rowcolor[gray]{0.9}
Number& DD\refstepcounter{datadefnum}\thedatadefnum \label{FlockingCohesion}:flockingCohesion\\
\hline
Label& The flocking cohesion at time t.\\
\hline
Symbol &$\mathbf{C}_\mathbf{t}$\\
\hline
SI Units & \si{\metre}\\
\hline
Equation& $\mathbf{C}_\mathbf{t} = 
\frac{1}{N} \sum_{i=1}^{N} \|(\mathbf{p}_{i,t} - \bar{\mathbf{p}}(\mathbf{t}))\|$\\
\hline
Description & 
It is one of the monitoring metrics of the flocking state, the flocking cohesion.\\
& $N$ is the number of agents,\\
& $\mathbf{p}_{i,t}$ is the agent \textit{i}'s position at time t(m).\\
& $\bar{\mathbf{p}}_\mathbf{t}$ is the center of flocking at time t(m).\\
\hline
Sources& \href{https://en.wikipedia.org/wiki/Flocking}{Definition of Flocking Cohesion}\\
\hline
Ref.\ By & \rref{R_FlockingStatus}\\
\hline
\end{tabular}
\end{minipage}\\

\subsubsection{Instance Models} \label{sec_instance}    

This section transforms the problem defined in Section~\ref{Sec_pd} into 
one which is expressed in mathematical terms. It uses concrete symbols defined 
in Section~\ref{sec_datadef} to replace the abstract symbols in the models 
identified in Sections~\ref{sec_theoretical} and~\ref{sec_gendef}.

The goals \gsref{GS_Simulation} and \gsref{GS_Comparison} are solved by 
\iref{ExplicitEuler}, \iref{RungeKutta4} or \iref{SemiImplicitEuler} depend on 
which and how many numerical integration methods selected by the user.

~\newline

\noindent
\begin{minipage}{\textwidth}
\renewcommand*{\arraystretch}{1.5}
\begin{tabular}{| p{\colAwidth} | p{\colBwidth}|}
\hline
\rowcolor[gray]{0.9}
Number& IM\refstepcounter{instnum}\theinstnum \label{TimeStepAcc}:timeStepAcceleration\\
\hline
Label& The acceleration of agent \textit{i} at time t\\
\hline
Input & $\mathbf{F}_{i,t}$, $\mathbf{m}$ \\
\hline
Output& $\mathbf{a}_{i,t}$ \\
\hline
Equation&
$\mathbf{a}_{i,t} = \frac{\mathbf{F}_{i,t}}{\mathbf{m}}$ \\
\hline
Description & 
The acceleration of agent \textit{i} at time t is computed using Newton's Second Law. \\
& $\mathbf{F}_{i,t}$ is the force acting on agent \textit{i} at time t, \\
& $\mathbf{m}$ is the mass of agent \textit{i}. For this system, each agent's \\
& mass is treated as a unit value(\aref{A_agentMass}).\\
\hline
Sources& -- \\
\hline
Ref.\ By & \iref{ExplicitEuler}, \iref{RungeKutta4}, \iref{SemiImplicitEuler}\\
\hline
\end{tabular}
\end{minipage}\\

~\newline

\noindent
\begin{minipage}{\textwidth}
\renewcommand*{\arraystretch}{1.5}
\begin{tabular}{| p{\colAwidth} | p{\colBwidth}|}
\hline
\rowcolor[gray]{0.9}
Number& IM\refstepcounter{instnum}\theinstnum \label{ExplicitEuler}:explicitEuler\\
\hline
Label& The Euler method used to calculate agent's status\\
\hline
Input & $\Delta t$, $\mathbf{p}_{i,t}$, $\mathbf{v}_{i,t}$, $\mathbf{a}_{i,t}$ \\
\hline
Output& $\mathbf{p}_{i,t+\Delta t}$, $\mathbf{v}_{i,t+\Delta t}$ \\
\hline
Equation&
$\mathbf{p}_{i,t+\Delta t} = \mathbf{p}_{i,t} + \Delta t \, \mathbf{v}_{i,t}, \quad
\mathbf{v}_{i,t+\Delta t} = \mathbf{v}_{i,t} + \Delta t \, \mathbf{a}_{i,t}$ \\
\hline
Description & 
The continuous-time system is approximated using an input time step $\Delta t$. \\
& $\mathbf{p}_{i,t}$ is the position of agent \textit{i} at time t, \\
& $\mathbf{v}_{i,t}$ is the velocity of agent \textit{i} at time t, \\
& $\mathbf{a}_{i,t}$ is the acceleration of agent \textit{i} at time t, \\
\hline
Sources& \href{https://en.wikipedia.org/wiki/Euler_method}{Euler Method} \\
\hline
Ref.\ By & \rref{R_MethodChoose}, \rref{R_AgentStatus}\\
\hline
\end{tabular}
\end{minipage}\\

~\newline

\noindent
\begin{minipage}{\textwidth}
\renewcommand*{\arraystretch}{1.5}
\begin{tabular}{| p{\colAwidth} | p{\colBwidth}|}
\hline
\rowcolor[gray]{0.9}
Number& IM\refstepcounter{instnum}\theinstnum \label{RungeKutta4}:rungeKutta4\\
\hline
Label& The Runge-Kutta 4 method used to calculate agent's status\\
\hline
Input & $\Delta t$, $\mathbf{p}_{i,t}$, $\mathbf{v}_{i,t}$, $\mathbf{a}_{i,t}$ \\
\hline
Output& $\mathbf{p}_{i,t+\Delta t}$, $\mathbf{v}_{i,t+\Delta t}$ \\
\hline
Equation &
$f(\mathbf{p}_i, \mathbf{v}_i) =\begin{bmatrix}\mathbf{v}_i \\\mathbf{a}_i\end{bmatrix}$, \\
& $\mathbf{k}_1 = f(\mathbf{p}_{i,t}, \mathbf{v}_{i,t})$, \\
& $\mathbf{k}_2 = f(\mathbf{p}_{i,t} + \frac{\Delta t}{2}\mathbf{k}_1, \mathbf{v}_{i,t} + \frac{\Delta t}{2}\mathbf{k}_1)$, \\
& $\mathbf{k}_3 = f(\mathbf{p}_{i,t} + \frac{\Delta t}{2}\mathbf{k}_2, \mathbf{v}_{i,t} + \frac{\Delta t}{2}\mathbf{k}_2)$, \\
& $\mathbf{k}_4 = f(\mathbf{p}_{i,t} + \Delta t \mathbf{k}_3, \mathbf{v}_{i,t} + \Delta t \mathbf{k}_3)$, \\
& $\mathbf{p}_{i,t+\Delta t} = \mathbf{p}_{i,t} + \frac{\Delta t}{6}(\mathbf{k}_1 + 2\mathbf{k}_2 + 2\mathbf{k}_3 + \mathbf{k}_4)$. \\
& $\mathbf{v}_{i,t+\Delta t} = \mathbf{v}_{i,t} + \frac{\Delta t}{6}(\mathbf{k}_1 + 2\mathbf{k}_2 + 2\mathbf{k}_3 + \mathbf{k}_4)$. \\
\hline
Description & 
The continuous-time system is approximated using an input time step $\Delta t$. \\
& $\mathbf{p}_{i,t}$ is the position of agent \textit{i} at time t, \\
& $\mathbf{v}_{i,t}$ is the velocity of agent \textit{i} at time t, \\
& $\mathbf{a}_{i,t}$ is the acceleration of agent \textit{i} at time t, \\
\hline
Sources& \href{https://en.wikipedia.org/wiki/Runge%E2%80%93Kutta_methods}{Runge-Kutta 4 Method} \\
\hline
Ref.\ By & \rref{R_MethodChoose}, \rref{R_AgentStatus}\\
\hline
\end{tabular}
\end{minipage}\\

~\newline

\noindent
\begin{minipage}{\textwidth}
\renewcommand*{\arraystretch}{1.5}
\begin{tabular}{| p{\colAwidth} | p{\colBwidth}|}
\hline
\rowcolor[gray]{0.9}
Number& IM\refstepcounter{instnum}\theinstnum \label{SemiImplicitEuler}:semiImplicitEuler\\
\hline
Label& The Semi-Implicit Euler used to calculate agent's status\\
\hline
Input & $\Delta t$, $\mathbf{p}_{i,t}$, $\mathbf{v}_{i,t}$, $\mathbf{a}_{i,t}$ \\
\hline
Output& $\mathbf{p}_{i,t+\Delta t}$, $\mathbf{v}_{i,t+\Delta t}$ \\
\hline
Equation&
$\mathbf{v}_{i,t+\Delta t} = \mathbf{v}_{i,t} + \Delta t \, \mathbf{a}_{i,t}, \quad
\mathbf{p}_{i,t+\Delta t} = \mathbf{p}_{i,t} + \Delta t \, \mathbf{v}_{i,t+\Delta t}$ \\
\hline
Description & 
The continuous-time system is approximated using an input time step $\Delta t$. \\
& $\mathbf{p}_{i,t}$ is the position of agent \textit{i} at time t, \\
& $\mathbf{v}_{i,t}$ is the velocity of agent \textit{i} at time t, \\
& $\mathbf{a}_{i,t}$ is the acceleration of agent \textit{i} at time t, \\
\hline
Sources& \href{https://en.wikipedia.org/wiki/Semi-implicit_Euler_method}{Semi-Implicit Euler Method} \\
\hline
Ref.\ By & \rref{R_MethodChoose}, \rref{R_AgentStatus}\\
\hline
\end{tabular}
\end{minipage}\\

\subsubsection{Input Data Constraints} \label{sec_DataConstraints}    

Table~\ref{TblInputVar} shows the data constraints on the input output
variables.  The column for physical constraints gives the physical limitations
on the range of values that can be taken by the variable.  The column for
software constraints restricts the range of inputs to reasonable values.  The
software constraints will be helpful in the design stage for picking suitable
algorithms.  The constraints are conservative, to give the user of the model the
flexibility to experiment with unusual situations.  The column of typical values
is intended to provide a feel for a common scenario.  The uncertainty column
provides an estimate of the confidence with which the physical quantities can be
measured.  This information would be part of the input if one were performing an
uncertainty quantification exercise.

The specification parameters in Table~\ref{TblInputVar} are listed in
Table~\ref{TblSpecParams}.

\begin{table}[!h]
  \caption{Input Variables} \label{TblInputVar}
  \renewcommand{\arraystretch}{1.2}
\noindent \begin{longtable*}{l l l l c} 
  \toprule
  \textbf{Var} & \textbf{Physical Constraints} & \textbf{Software Constraints} &
                             \textbf{Typical Value} & \textbf{Uncertainty}\\
  \midrule 
  $\mathbf{M}$ & - & $\mathbf{M} \in \{\text{EEuler}, \text{RK4}, \text{SIEuler}\}$ & $\text{SIEuler}$ & 0\\
  $N$ & $N \geq 0$ & $N^{\text{min}} \leq N \leq N^{\text{max}}$ & 100 & 0\\
  $\Delta t$ & $\Delta t > 0$ & ${\Delta t}^{\text{min}} \leq \Delta t \leq {\Delta t}^{\text{max}}$ & 0.01 & 0\\
  $k_a$ & $k_a \geq 0$ & $k^{\text{min}} \leq k_a \leq k^{\text{max}}$ & 0.5 & 0\\
  $k_c$ & $k_c \geq 0$ & $k^{\text{min}} \leq k_c \leq k^{\text{max}}$ & 0.5 & 0\\
  $k_s$ & $k_s \geq 0$ & $k^{\text{min}} \leq k_s \leq k^{\text{max}}$ & 0.5 & 0\\
  $k_o$ & $k_o \geq 0$ & $k^{\text{min}} \leq k_o \leq k^{\text{max}}$ & 0.5 & 0\\
  $k_g$ & $k_g \geq 0$ & $k^{\text{min}} \leq k_g \leq k^{\text{max}}$ & 0.5 & 0\\
  $\mathbf{p}_k$ & $\mathbf{p}_k \subset \mathbb{R}^2$ & $\mathbf{p}_k = (x, y), p^{\text{min}} < x, y < p^{\text{max}}$ & -- & 0\\
  $\mathbf{r}_k$ & $\mathbf{r}_k > 0$ & $\mathbf{r}^{\text{min}} \leq \mathbf{r}_k \leq \mathbf{r}^{\text{max}}$ & 50 & 0\\
  \bottomrule
\end{longtable*}
\end{table}

\begin{table}[!h]
\caption{Specification Parameter Values} \label{TblSpecParams}
\renewcommand{\arraystretch}{1.2}
\noindent \begin{longtable*}{l l} 
  \toprule
  \textbf{Var} & \textbf{Value} \\
  \midrule 
  $N^{\text{min}}$ & 10\\
  $N^{\text{max}}$ & 5000\\
  ${\Delta t}^{\text{min}}$ & 0.005 \si{\second}\\
  ${\Delta t}^{\text{max}}$ & 0.1 \si{\second}\\
  $k^{\text{min}}$ & 0\\
  $k^{\text{max}}$ & 2.0\\
  $p^{\text{min}}$ & -500 \si{\metre}\\
  $p^{\text{max}}$ & 1000 \si{\metre}\\
  $r^{\text{min}}$ & 1 \si{\metre}\\
  $r^{\text{max}}$ & 500 \si{\metre}\\
  \bottomrule
\end{longtable*}
\end{table}

\subsubsection{Properties of a Correct Solution} \label{sec_CorrectSolution}

The Table~\ref{TblOutputVar} shows the data constraints on the output variables. 
The column for physical constraints gives the physical limitations on the range 
of values that can be taken by the variable.

\begin{table}[!h]
\caption{Output Variables} \label{TblOutputVar}
\renewcommand{\arraystretch}{1.2}
\noindent \begin{longtable*}{l l} 
  \toprule
  \textbf{Var} & \textbf{Physical Constraints} \\
  \midrule 
  $\mathbf{p}_i$ & $\mathbf{p}_i \subset \mathbb{R}^2$ \\
  $\mathbf{v}_i$ & $\mathbf{v}_i \subset \mathbb{R}^2$ \\
  \bottomrule
\end{longtable*}
\end{table}

\section{Requirements}

This section provides the functional requirements, the business tasks that the
software is expected to complete, and the nonfunctional requirements, the
qualities that the software is expected to exhibit.

\subsection{Functional Requirements}

\noindent \begin{itemize}

\item[R\refstepcounter{reqnum}\thereqnum \label{R_Inputs}:] 
Input-Value: Input the values from Table~\ref{TabReqInputs}, which define the length of 
timestep, the parameters of flocking modules, the number of agents, the 
numerical integration method, and the layout of obstacles.

\item[R\refstepcounter{reqnum}\thereqnum \label{R_OutputInputs}:]
Input-Check: Check whether the inputs in \rref{R_Inputs} comply with the requirements in 
Subsection~\ref{sec_DataConstraints}.

\item[R\refstepcounter{reqnum}\thereqnum \label{R_MethodChoose}:]
Method-Choose-Support: Allow the user to choose the numerical integration method from the following 
list: \iref{ExplicitEuler}, \iref{RungeKutta4}, \iref{SemiImplicitEuler}.

\item[R\refstepcounter{reqnum}\thereqnum \label{R_AgentStatus}:]
Agents-Calculate-Values: Calculate all agents positions \textbf{p} and velocities \textbf{v} at each time 
step using the method selected in \rref{R_MethodChoose}.

\item[R\refstepcounter{reqnum}\thereqnum \label{R_FlockingStatus}:]
Flocking-Calculate-Values: Calculate the following values at each time step: $\bar{v}_\mathbf{t}$(\ddref{MeanSpeed}), 
$\mathbf{A}_\mathbf{t}$(\ddref{AlignMetric}) and $\mathbf{C}_\mathbf{t}$(\ddref{FlockingCohesion}).

\item[R\refstepcounter{reqnum}\thereqnum \label{R_AniamtionOutput}:]
Output-Animation: Use the data generated in \rref{R_AgentStatus} to output the flocking animation.

\item[R\refstepcounter{reqnum}\thereqnum \label{R_GraphOutput}:]
Output-Graphs: Use the data generated in \rref{R_FlockingStatus} to output line charts.

\end{itemize}

\noindent
\begin{table}[!h]
\caption{Specification Parameter Values} \label{TabReqInputs}
\renewcommand{\arraystretch}{1.2}
\noindent \begin{longtable*}{l l c} 
  \toprule
  \textbf{Symbol} & \textbf{Description} & \textbf{Units} \\
  \midrule 
  $N$ & The number of agent & --\\
  $\Delta t$ & The time step used by the numerical integration method & \si{\second}\\
  $k_a$ & The weight parameter of the alignment force & \si{\kilogram\per\second}\\
  $k_c$ & The weight parameter of the cohesion force & \si{\kilogram\per\second^2}\\
  $k_s$ & The weight parameter of the separation force & \si{\kilogram\meter^2\per\second^2}\\
  $k_o$ & The weight parameter of the obstacle avoidance force & \si{\kilogram\meter^2\per\second^2}\\
  $k_g$ & The weight parameter of the goal-seeking force & \si{\kilogram\meter\per\second^2}\\
  $\mathbf{p}_k$ & center position of obstacle \textit{k} & \si{\meter} \\
  $\mathbf{r}_k$ & center radius of obstacle \textit{k} & \si{\meter} \\
  \bottomrule
\end{longtable*}
\end{table}

~\newpage

\subsection{Nonfunctional Requirements}

This section provides the non-functional requirements, the qualities that the 
software is expected to exhibit.

\noindent \begin{itemize}

\item[NFR\refstepcounter{nfrnum}\thenfrnum \label{NFR_Correctness}:]
\textbf{Correctness}: The simulation shall generate flocking behavior that 
is perceived as realistic by at least 70\% of users.

\item[NFR\refstepcounter{nfrnum}\thenfrnum \label{NFR_Verifiability}:]
\textbf{Verifiability}: The code is tested with complete verification and 
validation plan.

\item[NFR\refstepcounter{nfrnum}\thenfrnum \label{NFR_Usability}:] 
\textbf{Usability}: The system shall provide a graphical user interface that 
allows users to visualize and modify simulation input parameters, such as the 
number of agents and obstacle layout. 

\item[NFR\refstepcounter{nfrnum}\thenfrnum \label{NFR_Maintainability}:]
\textbf{Maintainability}: If there is any change is made to the finished 
software, it will take at most 10\% of the original development time.

\end{itemize}

\subsection{Rationale}

This system treats each agent as having a unit mass ensures that the 
relationship between forces and accelerations is well-defined without 
introducing unnecessary physical complexity. Representing all obstacles as 
circular and defining their influence radius as twice the actual radius 
simplifies collision detection and neighborhood interactions while 
preserving realistic avoidance behavior. Fixing the goal position and 
generating starting positions randomly within a small circle allows consistent 
initialization of simulations while enabling variability in agent trajectories. 
Finally, assuming that all agents share identical intrinsic properties, 
including mass and motion dynamics, ensures homogeneity in the system, 
making the numerical integration of positions, velocities, and forces 
straightforward and reproducible, while the only source of variation arises 
from their initial positions.

\section{Likely Changes}    

This section lists the likely changes to be made to the software.

\noindent \begin{itemize}
\item[LC\refstepcounter{lcnum}\thelcnum\label{LC_ObsShapeChange}:]
\aref{A_obsShape} - Obstacles may not be limited to circular shapes.

\item[LC\refstepcounter{lcnum}\thelcnum\label{LC_InfluenceRadiusChange}:]
\aref{A_obsInfluence} - The influence radius of obstacles may change.

\item[LC\refstepcounter{lcnum}\thelcnum\label{LC_GoalChange}:]
\aref{A_goalPosition} - The agents' goal positions may be different.

\item[LC\refstepcounter{lcnum}\thelcnum\label{LC_StartAreaChange}:]
\aref{A_startPosition} - The initial generation area of agents may change.

\end{itemize}

\section{Unlikely Changes}    
This section lists the unlikely changes to be made to the software.
\noindent \begin{itemize}

\item[LC\refstepcounter{lcnum}\thelcnum\label{LC_NoMeanMass}:]
\aref{A_agentMass} - Agent mass does not play a meaningful role in the 
simulation results and is not expected to be modified.

\item[LC\refstepcounter{lcnum}\thelcnum\label{LC_agentHomogeneous}:]
\aref{A_similarAgent} - Supporting heterogeneous agents would significantly 
increase algorithmic complexity; therefore, agents are assumed to remain 
homogeneous.

\end{itemize}

\section{Traceability Matrices and Graphs}

The purpose of the traceability matrices is to provide easy references on what
has to be additionally modified if a certain component is changed.  Every time a
component is changed, the items in the column of that component that are marked
with an ``X'' may have to be modified as well.  Table~\ref{Table:trace} shows the
dependencies of theoretical models, general definitions, data definitions, and
instance models with each other. Table~\ref{Table:R_trace} shows the
dependencies of instance models, requirements, and data constraints on each
other. Table~\ref{Table:A_trace} shows the dependencies of theoretical models,
general definitions, data definitions, instance models, and likely changes on
the assumptions.

Columns represent items that trace to source items listed in rows.

\afterpage{
% \begin{landscape}
\begin{table}[h!]
\centering
\begin{tabular}{|c|c|c|c|c|c|c|c|c|c|c|c|c|c|c|c|c|c|c|c|}
\hline
	& \aref{A_agentMass}& \aref{A_obsShape}& \aref{A_obsInfluence}
  & \aref{A_goalPosition}& \aref{A_startPosition}& \aref{A_similarAgent} \\
\hline
\tref{NewtonSecLaw}              & & & & & & \\ \hline
\tref{ConTimeDy}                 & & & & & & \\ \hline
\dref{ForceCombination}          & & & & & & \\ \hline
\ddref{SingleObstacle}           & &X& & & & \\ \hline
\ddref{ObstacleSet}              & & & & & & \\ \hline
\ddref{StartPosition}            & & & & & & \\ \hline
\ddref{AlignForce}               & & & & & & \\ \hline
\ddref{CohereForce}              & & & & & & \\ \hline
\ddref{SeparateForce}            & & & & & & \\ \hline
\ddref{ObsAvoidForce}            & & &X& & & \\ \hline
\ddref{GoalSeekForce}            & & & &X& & \\ \hline
\ddref{MeanSpeed}                & & & & & & \\ \hline
\ddref{AlignMetric}              & & & & & & \\ \hline
\ddref{FlockingCenter}           & & & & & & \\ \hline
\ddref{FlockingCohesion}         & & & & & & \\ \hline
\iref{TimeStepAcc}               & & & & & & \\ \hline
\iref{ExplicitEuler}             & & & & & & \\ \hline
\iref{RungeKutta4}               & & & & & & \\ \hline
\iref{SemiImplicitEuler}         & & & & & & \\ \hline
\lcref{LC_ObsShapeChange}        & &X& & & & \\ \hline
\lcref{LC_InfluenceRadiusChange} & & &X& & & \\ \hline
\lcref{LC_GoalChange}            & & & &X& & \\ \hline
\lcref{LC_StartAreaChange}       & & & & &X& \\ \hline
\lcref{LC_NoMeanMass}            &X& & & & & \\ \hline
\lcref{LC_agentHomogeneous}      & & & & & &X\\
\hline
\end{tabular}
\caption{Traceability Matrix Showing the Connections Between Assumptions and Other Items}
\label{Table:A_trace}
\end{table}
% \end{landscape}
}

\afterpage{
\begin{landscape}
\begin{table}[h!]
\centering
\begin{tabular}{|c|c|c|c|c|c|c|c|c|c|c|c|c|c|c|c|c|c|c|c|}
\hline        
	& \tref{NewtonSecLaw}& \tref{ConTimeDy}& \dref{ForceCombination}
  & \ddref{SingleObstacle}& \ddref{ObstacleSet} & \ddref{StartPosition}
  & \ddref{AlignForce} & \ddref{CohereForce} & \ddref{SeparateForce} 
  & \ddref{ObsAvoidForce} & \ddref{GoalSeekForce} 
  & \ddref{FlockingCenter} & \ddref{FlockingCohesion} 
  & \iref{TimeStepAcc}& \iref{ExplicitEuler}& \iref{RungeKutta4}& \iref{SemiImplicitEuler} \\
\hline
\tref{NewtonSecLaw}       & & & & & & & & & & & & & & & & & \\ \hline
\tref{ConTimeDy}          & & & & & & & & & & & & & & & & & \\ \hline
\dref{ForceCombination}   & & & & & & &X&X&X&X&X& & & & & & \\ \hline
\ddref{SingleObstacle}    & & & & & & & & & & & & & & & & & \\ \hline
\ddref{ObstacleSet}       & & & &X& & & & & & & & & & & & & \\ \hline
\ddref{StartPosition}     & & & & & & & & & & & & & & & & & \\ \hline
\ddref{AlignForce}        & & & & & & & & & & & & & & & & & \\ \hline
\ddref{CohereForce}       & & & & & & & & & & & & & & & & & \\ \hline
\ddref{SeparateForce}     & & & & & & & & & & & & & & & & & \\ \hline
\ddref{ObsAvoidForce}     & & & &X&X& & & & & & & & & & & & \\ \hline
\ddref{GoalSeekForce}     & & & & & & & & & & & & & & & & & \\ \hline
\ddref{FlockingCenter}    & & & & & & & & & & & & & & & & & \\ \hline
\ddref{FlockingCohesion}  & & & & & & & & & & & &X& & & & & \\ \hline
\iref{TimeStepAcc}        &X& &X& & & & & & & & & & & & & & \\ \hline
\iref{ExplicitEuler}      & &X& & & &X& & & & & & & &X& & & \\ \hline
\iref{RungeKutta4}        & &X& & & &X& & & & & & & &X& & & \\ \hline
\iref{SemiImplicitEuler}  & &X& & & &X& & & & & & & &X& & & \\
\hline
\end{tabular}
\caption{Traceability Matrix Showing the Connections Between Items of Different Sections}
\label{Table:trace}
\end{table}
\end{landscape}
}

\begin{table}[h!]
\centering
\begin{tabular}{|c|c|c|c|c|c|c|c|c|c|c|c|c|c|c|c|}
\hline
	& \ddref{MeanSpeed}& \ddref{AlignMetric}& \ddref{FlockingCohesion}
	& \iref{TimeStepAcc}& \iref{ExplicitEuler}& \iref{RungeKutta4}& \iref{SemiImplicitEuler}
  & \rref{R_Inputs}& \rref{R_OutputInputs}& \rref{R_MethodChoose}& \rref{R_AgentStatus}
  & \rref{R_FlockingStatus}& \rref{R_AniamtionOutput}& \rref{R_GraphOutput} \\
\hline
\ddref{MeanSpeed}            & & & & & & & & & & & & & & \\ \hline
\ddref{AlignMetric}          & & & & & & & & & & & & & & \\ \hline
\ddref{FlockingCohesion}     & & & & & & & & & & & & & & \\ \hline
\iref{TimeStepAcc}           & & & & & & & & & & & & & & \\ \hline
\iref{ExplicitEuler}         & & & & & & & & & & & & & & \\ \hline
\iref{RungeKutta4}           & & & & & & & & & & & & & & \\ \hline
\iref{SemiImplicitEuler}     & & & & & & & & & & & & & & \\ \hline
\rref{R_Inputs}              & & & & & & & & & & & & & & \\ \hline
\rref{R_OutputInputs}        & & & & & & & &X& & & & & & \\ \hline
\rref{R_MethodChoose}        & & & & &X&X&X& & & & & & & \\ \hline
\rref{R_AgentStatus}         & & & & &X&X&X& & &X& & & & \\ \hline
\rref{R_FlockingStatus}      &X&X&X& & & & & & & & & & & \\ \hline 
\rref{R_AniamtionOutput}     & & & & & & & & & & &X& & & \\ \hline
\rref{R_GraphOutput}         & & & & & & & & & & & &X& & \\ 
\hline
\end{tabular}
\caption{Traceability Matrix Showing the Connections Between Requirements and Instance Models}
\label{Table:R_trace}
\end{table}

% The purpose of the traceability graphs is also to provide easy references on
% what has to be additionally modified if a certain component is changed.  The
% arrows in the graphs represent dependencies. The component at the tail of an
% arrow is depended on by the component at the head of that arrow. Therefore, if a
% component is changed, the components that it points to should also be
% changed. Figure~\ref{Fig_ATrace} shows the dependencies of theoretical models,
% general definitions, data definitions, instance models, likely changes, and
% assumptions on each other. Figure~\ref{Fig_RTrace} shows the dependencies of
% instance models, requirements, and data constraints on each other.

% \begin{figure}[h!]
% 	\begin{center}
% 		%\rotatebox{-90}
% 		{
% 			\includegraphics[width=\textwidth]{ATrace.png}
% 		}
% 		\caption{\label{Fig_ATrace} Traceability Matrix Showing the Connections Between Items of Different Sections}
% 	\end{center}
% \end{figure}


% \begin{figure}[h!]
% 	\begin{center}
% 		%\rotatebox{-90}
% 		{
% 			\includegraphics[width=0.7\textwidth]{RTrace.png}
% 		}
% 		\caption{\label{Fig_RTrace} Traceability Matrix Showing the Connections Between Requirements, Instance Models, and Data Constraints}
% 	\end{center}
% \end{figure}

% \section{Development Plan}

% \plt{This section is optional.  It is used to explain the plan for developing
%   the software.  In particular, this section gives a list of the order in which
%   the requirements will be implemented.  In the context of a course  this is
%   where you can indicate which requirements will be implemented as part of the
%   course, and which will be ``faked'' as future work.  This section can be
%   organized as a prioritized list of requirements, or it could should the
%   requirements that will be implemented for ``phase 1'', ``phase 2'', etc.}

\section{Values of Auxiliary Constants}

All auxiliary constants used in this model are listed in Table~\ref{TblSpecParams} 
(Section ~\ref{sec_DataConstraints}).


\newpage

\bibliographystyle {plainnat}
\bibliography {../../refs/References}

% \newpage

% \noindent \plt{The following is not part of the template, just some things to consider
%   when filing in the template.}

% \noindent \plt{Grammar, flow and \LaTeX advice:
% \begin{itemize}
% \item For Mac users \texttt{*.DS\_Store} should be in \texttt{.gitignore}
% \item \LaTeX{} and formatting rules
% \begin{itemize}
% \item Variables are italic, everything else not, includes subscripts (link to
%   document)
% \begin{itemize}
% \item \href{https://physics.nist.gov/cuu/pdf/typefaces.pdf}{Conventions}
% \item Watch out for implied multiplication
% \end{itemize}
% \item Use BibTeX
% \item Use cross-referencing
% \end{itemize}
% \item Grammar and writing rules
% \begin{itemize}
% \item Acronyms expanded on first usage (not just in table of acronyms)
% \item ``In order to'' should be ``to''
% \end{itemize}
% \end{itemize}}

% \noindent \plt{Advice on using the template:
% \begin{itemize}
% \item Difference between physical and software constraints
% \item Properties of a correct solution means \emph{additional} properties, not
%   a restating of the requirements (may be ``not applicable'' for your problem).
%   If you have a table of output constraints, then these are properties of a
%   correct solution.
% \item Assumptions have to be invoked somewhere
% \item ``Referenced by'' implies that there is an explicit reference
% \item Think of traceability matrix, list of assumption invocations and list of
%   reference by fields as automatically generatable
% \item If you say the format of the output (plot, table etc), then your
%   requirement could be more abstract
% \end{itemize}
% }

% \newpage{}
% \section*{Appendix --- Reflection}

% \wss{Not required for CAS 741}

% The information in this section will be used to evaluate the team members on the
% graduate attribute of Lifelong Learning.  

% \input{../Reflection.text}

% \input{../SRS_Reflection.text}

\end{document}