\documentclass{article}

\usepackage{tabularx}
\usepackage{booktabs}

\title{Problem Statement and Goals\\\ Numerical Integration of Flocking Dynamics over Structured Terrains}

\author{\ Yibing Mei}

\date{}

\input{../Comments.text}
\input{../Common.text}

\begin{document}

\maketitle

\begin{table}[hp]
\caption{Revision History} \label{TblRevisionHistory}
\begin{tabularx}{\textwidth}{llX}
\toprule
\textbf{Date} & \textbf{Developer(s)} & \textbf{Change}\\
\midrule
2026/01/16 & Yibing Mei & Initial upload\\
\bottomrule
\end{tabularx}
\end{table}

\section{Problem Statement}

\subsection{Problem}

This project simulates flocking dynamics over structured terrains with features such as obstacles and narrow passages. 
It compares multiple numerical integration methods to evaluate their effects on stability, coherence, long-term behavior, 
and the success rate of passing complex environments. The results are visualized through animations to support analysis 
and help identify methods that better meet the needs of the relevant stakeholders, such as greater group cohesion.


\subsection{Inputs and Outputs}

Input: Users can configure key simulation parameters, including the number of agents, the numerical integration method, 
and the layout of obstacles. \\
Output: The simulation generates animations of agent-based flocking behavior over the structured terrain. In addition, 
it produces time-series curves for stability, coherence, and group cohesion throughout the simulation, as well as the 
final rate of passing through the terrain for all agents.


\subsection{Stakeholders}

The primary stakeholders of this project are researchers and students in scientific computing and applied mathematics who 
use numerical simulations to study collective dynamics. The results may also be of interest to researchers working on 
flocking, swarm dynamics, and multi-agent systems, such as those in swarm robotics, particularly in the simulations 
involving complex environments. Additionally, those working in biological collective behavior modeling, traffic flow, 
environmental and geographic modeling, and crowd animation may require large-scale agent-based simulation results for 
their studies.


\subsection{Environment}

The simulation is designed to run on a standard desktop or laptop environment. Users should have a system with Windows 
10 or 11, and interaction with the simulation, such as adjusting parameters and controlling the visualization, is 
performed via keyboard and mouse.


\section{Goals}

\subsection{Implement multiple numerical integration methods}
Implement commonly used methods such as Euler, Runge-Kutta, etc.

\subsection{Enable user-defined terrains}
Allow users to configure obstacles and terrain features, and simulate agent flocking behavior.

\subsection{Compute and visualize key metrics over time}
Calculate stability, coherence, and group cohesion in real time for each integration method, generate their time-series curves, 
and compute the final navigation success rate.

\subsection{Support comparison of multiple methods}
Allow users to run simulations with different methods on the same terrain and generate comparative metric curves.

\subsection{Export the simulation as animation}
Support exporting the flocking process into an animation file for further analysis or presentation.


\section{Stretch Goals}

\subsection{Support real-time animation of the simulation}
Display agent flocking behavior dynamically during the simulation for interactive observation.

\subsection{Handle very large numbers of agents}
Enable simulation with tens of thousands of agents efficiently.

\subsection{Extend simulation from 2D to 3D}
Allow agents to navigate and visualize flocking behavior in 3D terrains.

\subsection{Enable tracking of individual agents}
Support focusing on and observing the behavior of a single agent within the flock.


\section{Extras}

This project is a research project. \\
The goal is to identify numerical integration methods that lead to more stable and coherent flocking behavior over 
structured terrains. By implementing and comparing those methods under non-standard environmental settings, the project 
aims to analyze differences in flocking dynamics that are not known a priori, rather than merely reproducing a fixed 
simulation outcome.

\end{document}